

\section{Eigenschaften des Verhandlungsführers}

\note[itemize]{
  \item note 1
  \item note test 2
}

\begin{frame}{Verhandlungen}
  "More of an art than science" \cite[][p.12]{mccarthy_advanced_2015}

  % TODO
	% -integrative Verhandlungen
    % -win-win vs. win-lose
    % Verhandlung: Auflösung eines Konflikts

\end{frame}

\subsection{Einfühlungsvermögen}

\begin{frame}
  % TODO picture: Geiselnahme
  "Steige nie mit rationalen Argumenten ein" \cite{obermeier_karrieresprung_nodate}
\end{frame}

\begin{frame}
  % TODO picture: Eisberg
  \begin{itemize}
    \item Motive, Bedürfnisse hinter den Forderungen des Gegenübers verstehen \cites{mccarthy_advanced_2015}[p. 106]{wannenwetsch_erfolgreicher_2009}
  \item Analysieren \cite[][4.4.2]{helmold_verhandlungskonzepte_2019} \begin{itemize}
	  \item Primärmotive: lebensnotwendig, z.B. Hunger, Geborgenheit
	  \item Sekundärmotive: angelernt, z.B. Anerkennung, Erfolg, Macht, Reichtum
	\end{itemize}
  \item Irrationalität im Verhalten \cite[][p. 12]{mccarthy_advanced_2015}
  \item Sachliche Argumente können unbedeutend werden 
  \end{itemize}
	  

\end{frame}


\begin{frame}{Empathie vs. Perspektive beziehen} % TODO besseres Wort
  Was ist effektiver?

  % TODO quote format
  Perspektive beziehen: "Es gibt immer zwei Seiten der Medaille."
  Empathie: "Manchmal tut es mir gar nicht so leid, wenn es Leuten schlecht geht."

  Die Welt aus der Sicht des Anderen scheint zu effizienteren Abmachungen mit größerem kollektiven und individuellen Zugewinn zu führen \cite{galinsky_why_2008}.

  \note[itemize]{
    \item can be applicable in certain negoations:
    \item angry negotatiors with desire to be heard
    \item calming, softening, de-escalation
    \item helpful in mediation
  }
  
  Was helfen kann: Behavioristische Mimikry \cite{thompson_negotiation_2010}[p. 500]
  \note[itemize]{
     \item people tend to engage in face, rubbin, foot shaking, smiling more in presence of other in that behavior
     \item behavioral mimicry increases liking and rapport
     \item mood contagion
     \item expression of liking: dominance complementaritiy: resp.\ with submissive and vice versa
   }
\end{frame}

\begin{frame}{"Der erste Eindruck zählt"}
  Mentales Modell: kognitive Repräsentation der erwarteten Verhandlung \cite[][p. 287ff]{bazerman_negotiation_2000}
  \begin{itemize}
	\item Beziehungen
	\item Attributionen
	  \note[itemize]{
		\item konstante Musterkennung
		\item Zuordnung zu einer Schublade, zu Rollen (Geschäftspartner, Freund) 
		\item ohne Verbindung zur Realität
	  }
	\item Struktur und Prozess
  \end{itemize}
  
	  Individuelle Mentale Modelle können zu Beginn einer Verhandlung häufig Feindlichkeit erwarten.
      Erst über den Lauf der Verhandlung wird über Win-Win-Potenziale gelernt. Wenn sich die Wahrnehmung ändert, dann nur sehr früh zu Beginn \cite{thompson_social_1990}
	  \cite[][p. 287ff]{bazerman_negotiation_2000}. 

\end{frame}



\begin{frame}{Persönlichkeitstypen}
  % TODO picture: Persönlichkeitstypen
  Muss ins Bewusstsein gerückt werden und beim Gegenüber mit einbezogen werden
	\cite[][4.5.5]{helmold_verhandlungskonzepte_2019}
  \begin{itemize}
	\item "Der Schlaue": an das Ego appellieren
	\item "Der Träge": direkt nach der Meinung fragen
	\item "Der Zurückhaltende": über Fragen einbinden
  \end{itemize}

  Myers-Briggs-Typenindikator:
  \begin{itemize}
	\item Sensing-Thinking: klare Fakten, gute Argumente
	\item Intuition-Feeling: Vertrauen aufbauen, Respekt ausstrahlen
  \end{itemize}
    
\end{frame}


\subsection{Respekt}

\begin{frame}

  Respekt\ldots

  \begin{itemize}
    \item[\ldots]gegenüber dem Verhandlungspartner: Grenzen kennen
    \item[\ldots]gegenüber sich selbst: Selbstbewusstsein
    \item[\ldots]ausstrahlen: Körpersprache
  \end{itemize}

\end{frame}

\begin{frame}{Respekt zollen}

  Zuerst eine Beziehung aufbauen, dann Fakten schaffen \cite{mccarthy_advanced_2015}[p. 16]
  
  Vertrauen schaffen, um mentale Hürden zu senken und den Informationsaustausch zu fördern \note{-> Integrativ}

  Aufmerksames, aktives Zuhören (siehe \hyperref[sec:geduld]{Geduld})

  Pragmatisch sein: persönliche Emotionen wahrnehmen, aber nicht immer zeigen

  Notizen nehmen \note{Erinnerungen, Aufmerksamkeit zeigen}

\end{frame}

\begin{frame}{Selbstbewusstsein}

  Propriozeptive Psychology: "Fake it \'til you make it" \cite{mccarthy_advanced_2015}[p. 38]

  \note{Experiment: Leute for dem Computer bildschirm – horizontal und vertikal bewegende Teile}
  \note{people smile when they're happy, but they're also happier because they smile}

  Indirekt gewinnen über \ldots 

\end{frame}

\begin{frame}{Ausstrahlung}

  Kongruentes Verhalten ist glaubwürdig: Körper, Stimme und Inhalt \cite{wannenwetsch_erfolgreicher_2009}

  93\% der Wirkung sind unabhängig vom Inhalt \cite{wannenwetsch_erfolgreicher_2009}

  \note{können auch negative Ausdrücke sien: bei unrealistischen Forderungen}

  Säulen:
  \begin{itemize}
    \item Mimik und Gestik
    \item Körperhaltung
    \item Blickkontakt
    \item Kleidung: ordentlich, sauber, dezent, passend \note{auch zur Persönlihckeit}
    \item Stimme: Verständlichkeit und Moudulation \note{Spannung erzeugen, Positionen, Prioritäten hervorheben, aber keine Lautstärke = Dominanz}
  \end{itemize}

  %TODO pictures HelmoldNonVerb


\end{frame}
