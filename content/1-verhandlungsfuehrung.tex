
% TODO wannenwetsch zitate

\section{Eigenschaften des Verhandlungsführers}

\note[itemize]{
  \item note 1
  \item note test 2
}

\begin{frame}{Verhandlungen}
  "More of an art than science" \cite[][p.12]{mccarthy_advanced_2015}

  % TODO
	% -integrative Verhandlungen
    % -win-win vs. win-lose
    % Verhandlung: Auflösung eines Konflikts
    % Nash equilibria
    % joint gains

\end{frame}

\subsection{Einfühlungsvermögen}

\begin{frame}
  % TODO picture: Geiselnahme
  "Steige nie mit rationalen Argumenten ein" \cite{obermeier_karrieresprung_nodate}
\end{frame}

\begin{frame}
  % TODO picture: Eisberg
  \begin{itemize}
    \item Motive, Bedürfnisse hinter den Forderungen des Gegenübers verstehen \cites[][p. 106]{mccarthy_advanced_2015}{wannenwetsch_erfolgreicher_2009}
  \item Analysieren \cite[][ch. 4.4.2]{helmold_verhandlungskonzepte_2019} \begin{itemize}
	  \item Primärmotive: lebensnotwendig, z.B. Hunger, Geborgenheit
	  \item Sekundärmotive: angelernt, z.B. Anerkennung, Erfolg, Reichtum
	\end{itemize}
  \item Irrationalität im Verhalten \cite[][p. 12]{mccarthy_advanced_2015}
  \item Sachliche Argumente können unbedeutend werden 
  \end{itemize}
	  

\end{frame}


\begin{frame}{Empathie vs. Hineinversetzen}
  Was ist effektiver?

  % TODO quote format
  Hineinversetzen: \enquote{Es gibt immer zwei Seiten der Medaille.}
  Empathie: \enquote{Manchmal tut es mir gar nicht so leid, wenn es Leuten schlecht geht.}

  Das Hineinversetzen ermöglicht effizientere Abmachungen mit größerem kollektiven und individuellen Zugewinn \cite{galinsky_why_2008}.

  \note[itemize]{
    \item can be applicable in certain negoations:
    \item angry negotatiors with desire to be heard
    \item calming, softening, de-escalation
    \item helpful in mediation
  }
  
  Was helfen kann: Behavioristische Mimikry \cite[][p. 500]{thompson_negotiation_2010}
  \note[itemize]{
     \item people tend to engage in face, rubbin, foot shaking, smiling more in presence of other in that behavior
     \item behavioral mimicry increases liking and rapport
     \item mood contagion
     \item expression of liking: dominance complementaritiy: resp.\ with submissive and vice versa
   }
\end{frame}

\begin{frame}{"Der erste Eindruck zählt"}
  Mentales Modell: kognitive Repräsentation der erwarteten Verhandlung \cite[][p. 287ff]{bazerman_negotiation_2000}
  \begin{itemize}
	\item Beziehungen
	\item Attributionen
	  \note[itemize]{
		\item konstante Musterkennung
		\item Zuordnung zu einer Schublade, zu Rollen (Geschäftspartner, Freund) 
		\item ohne Verbindung zur Realität
	  }
	\item Struktur und Prozess
  \end{itemize}
  
	  Individuelle Mentale Modelle können zu Beginn einer Verhandlung häufig Feindlichkeit erwarten.
      Erst über den Lauf der Verhandlung wird über Win-Win-Potenziale gelernt. Wenn sich die Wahrnehmung ändert, dann nur sehr früh zu Beginn \cite{thompson_social_1990}
	  \cite[][p. 287ff]{bazerman_negotiation_2000}. 

\end{frame}


\begin{frame}{Persönlichkeitstypen}
  % TODO picture: Persönlichkeitstypen
  Muss ins Bewusstsein gerückt werden und beim Gegenüber miteinbezogen werden
    \cite[][ch. 4.5.5]{helmold_verhandlungskonzepte_2019}.
  \begin{itemize}
	\item "Der Schlaue": an das Ego appellieren
	\item "Der Träge": direkt nach der Meinung fragen
	\item "Der Zurückhaltende": über Fragen einbinden
  \end{itemize}

  Myers-Briggs-Typenindikator:
  \begin{itemize}
    \item \textit{Sensing-Thinking}: klare Fakten, gute Argumente
    \item \textit{Intuition-Feeling}: Vertrauen aufbauen, Respekt ausstrahlen
  \end{itemize}
    
\end{frame}


\subsection{Respekt}

\begin{frame}

  Respekt \ldots

  \begin{itemize}
    \item[\ldots] gegenüber dem Verhandlungspartner: Grenzen kennen
    \item[\ldots] gegenüber sich selbst: Selbstbewusstsein
    \item[\ldots] ausstrahlen: Körpersprache
  \end{itemize}

\end{frame}

\begin{frame}{Respekt zollen}

  \begin{itemize}
    \item Zuerst eine Beziehung aufbauen, dann Fakten schaffen \cite[][p. 16]{mccarthy_advanced_2015}
    \item Vertrauen schaffen, um mentale Hürden zu senken und den Informationsaustausch zu fördern \note{-> Integrativ}
    \item Aufmerksames, aktives Zuhören (siehe \hyperref[sec:geduld]{Geduld})
    \item Notizen nehmen \note{Erinnerungen, Aufmerksamkeit zeigen}
  \end{itemize}

\end{frame}

\begin{frame}{Selbstbewusstsein}

  % TODO passendes bild?
  Propriozeptive Psychologie: \enquote{Fake it `til you make it} \cite[][p. 38]{mccarthy_advanced_2015}

  \note{Experiment: Leute for dem Computer bildschirm – horizontal und vertikal bewegende Teile}
  \note{people smile when they're happy, but they're also happier because they smile}

  Kann also indirekt gewonnen werden über \ldots 

\end{frame}

\begin{frame}{Ausstrahlung}

  Kongruentes Verhalten ist glaubwürdig: Körper, Stimme und Inhalt müssen sich decken \cite{wannenwetsch_erfolgreicher_2009}.

  93\% der Wirkung sind unabhängig vom Inhalt \cite{wannenwetsch_erfolgreicher_2009}.

  \note{können auch negative Ausdrücke sien: bei unrealistischen Forderungen}

  Säulen:
  \begin{itemize}
    \item Mimik und Gestik \note{kann schwer festgemacht werden}
    \item Körperhaltung
    \item Blickkontakt \note{kein Starren}
    \item Kleidung: ordentlich, sauber, dezent, passend \note{auch zur Persönlihckeit}
    \item Stimme: Verständlichkeit und Modulation \note{Spannung erzeugen, Positionen, Prioritäten hervorheben, aber keine Lautstärke = Dominanz}
  \end{itemize}

  %TODO pictures HelmoldNonVerb

\end{frame}


\subsection{Integrität}

\begin{frame}

  Täuschungen\ldots 
  \begin{itemize}
    \item[\ldots]passieren häufiger, wenn mehr auf dem Spiel steht \cites[][p. 292]{bazerman_negotiation_2000}[][p. 501]{thompson_negotiation_2010}
    \item[\ldots]zerstört Vertrauen und verringert damit die Chancen auf integrative Lösungen % TODO definition
    \item[\ldots]kommt von einer Kombination aus Versuchung, Unsicherheit, Machtmangel und Anonymität von Opfern \cite[][p. 501]{thompson_negotiation_2010}
  \end{itemize}

\end{frame}

\begin{frame}
  Gesellschaftiche Regeln sollten eingehalten werden:

  \begin{itemize}
    \item Ehrlichkeit
    \item Zuverlässigkeit
    \item Pünktlichkeit
    \item Legalität
    \item keine Machtinstrumentalisierung
    \item Fairness
    \item Verantwortungsbewusstsein \note{Entschludigung, Formalitäten, Dokumente, Termine}
  \end{itemize}

  Freunde mit kommunaler Orientierung sind am wahrscheinlichsten dazu in der Lage beidseitige Zugewinne zu sichern \cite[][p. 502]{thompson_negotiation_2010}.
  \note{ohne Erwartungen auf Rückzahlung <-> kommunale Orientierung: Austausch-Orientierung, aufhebende Gegenseitigkeit von wohlwollenden Handlungen}
  
  Pragmatisch sein: persönliche Emotionen wahrnehmen, aber nicht immer zeigen. Sie können einen negativen Effekt auf Verhandlungen haben. \cite[][p. 13]{mccarthy_advanced_2015}
\end{frame}

\begin{frame}{Fairness}

  Eine Definition kann vom Kulturkreis und der Vergangenheit abhängen \cite{rabin_incorporating_1993}. \note{vergangener Preis}
  
  Altruismus motiviert Altruismus, Hass befördert Hass. Für beides sind Menschen bereit Materielles aufzugeben. Je niedriger die materiellen Kosten, desto größer diese Effekte. \cite{rabin_incorporating_1993}

  Wünsche und Bedürfnisse aller sollten berücksichtigt werden. Doch \textquote[\cite{obermeier_karrieresprung_nodate}]{es geht nicht um reale, sondern gefühlsmäßige Win-Win Situationen.} 

\end{frame}

\subsection{Geduld} \label{sec:geduld}

\begin{frame}

  Positives Denken \cite[][p. 33ff]{mccarthy_advanced_2015}
  \begin{itemize}
    \item die Einstellung bestimmt das Verhalten
    \item Konflikte als Möglichkeit, statt Problem
    \item inhärentes Vertrauen
    \item schafft Motivation und Geduld
  \end{itemize}

  % TODO picture: Brick Wall?
  Zweite Versuche sind meist notwendig und Teil des Wegs zum Erfolg. 

\end{frame}

\begin{frame}

  Eine Form Geduld zu zeigen: Aktives Zuhören
  
  \begin{itemize}
    \item animiert zum Weiterreden
    \item ist konzentriertes, eingehendes Zuhören
    \item beinhaltet Feedback und Verständnis-Fragen: Johari-Fenster \cite[][p. 97]{mccarthy_advanced_2015} \note{gegenseitig Blind Spots entfernen}
    \item beachtet auch Betonung und Körpersprache
    \item zollt Respekt
  \end{itemize}
    
  \enquote{Schweigen ist Gold}. Nach überzeugenden Argumenten und wenn man nichts zu sagen hat sollte man schweigen \cites{obermeier_karrieresprung_nodate}[][ch. 4.5.2]{helmold_verhandlungskonzepte_2019}.

\end{frame}


\subsection{Humor und Kreativität}

\begin{frame}
  Humor: die Fähigkeit auch in schwierigen Situationen Amüsantes zu erkennen. % TODO def frame

  Humor hilft der Kreativität. Diese besteht aus Spontanität, Phantasie und Originalität und schafft Handlungsspielräume und damit Möglichkeiten zur Einigung \cite{wannenwetsch_erfolgreicher_2009}.

\end{frame}

\begin{frame}

  Mentale Vorbereitung kann helfen \cite{wannenwetsch_erfolgreicher_2009}

  \begin{itemize}
    \item Propriozeptive Psychologie \cite[][p. 37ff]{mccarthy_advanced_2015}
    \item Befreiung von Zwängen
    \item Einstellung: Interesse, Neugier oder Langeweile, Desinteresse?
  \end{itemize}

\end{frame}


\subsection{Selbstdisziplin und Durchhaltevermögen}

\begin{frame}{Selbstdisziplin}

  Eigen-Motivation, Autarkie, Selbstbestimmtheit

  kann gefördert werden durch\ldots
  \begin{itemize}
    \item[\ldots]Fokus auf Stärken, statt Schwächen \note{mehr Spaß, inhärenter Antrieb}
    \item[\ldots]Kontrolle über Emotionen: Zorn und Begeisterung ggü. Offenheit und Interesse
    \item[\ldots]Vorstellung vom Erfolg \note{am Strand}
    \item[\ldots]Zelebrieren des Erfolgs \note{feiern!}
  \end{itemize}

  \ldots und fördert das Durchhältvermögen\ldots

\end{frame}

\begin{frame}{Durchhaltevermögen}

Ohne guten Grund nachzugeben kann Misstrauen fördern. Es sollten nur dann Zugeständnisse gemacht werden, wo sie von Nutzen sind (Gegenleistungen). \cites{obermeier_karrieresprung_nodate}[ch. 4.5.2]{helmold_verhandlungskonzepte_2019}.

  Es sollte stets versucht werden, die Argumente der Gegenseite zu entkräftigen. Man darf nach Evidenz verlangen. Der Fokus sollte auf den schwächsten Argumenten der Gegenseite liegen \cite[][ch. 4.5.2]{helmold_verhandlungskonzepte_2019}.

\end{frame}

\begin{frame}

  \ldots und alles baut auf dem Fundament:
  
  Gesunde Ernährung, Schlaf und Work-Life Balance.

\end{frame}

