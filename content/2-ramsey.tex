%!TEX root = ../../main.tex

\begin{frame}[plain]
	\section{Satz 2: Ramsey-Zahlen}
\end{frame}

\nocite{AmericanMathematicalSociety1979, Alon2008}

\begin{frame}[fragile]{Gute und schlechte Parties}

\textcolor{mLightBrown}{Freunde}

\textcolor{mLightGreen}{Fremde}

\begin{columns}[T,onlytextwidth]

\column{0.25\textwidth}

\begin{dot2tex}[neato,mathmode,graphstyle={scale=0.7}]
 graph k3none {
    a -- b [color=mLightGreen];
    b -- c [color=mLightGreen];
    a -- c [color=mLightGreen];
  }
\end{dot2tex}  

\column{0.25\textwidth}
\begin{dot2tex}[neato,mathmode,graphstyle={scale=0.7}]
 graph k3one {
    a -- b [color=mLightBrown];
    b -- c [color=mLightGreen];
    a -- c [color=mLightGreen];
  }
\end{dot2tex}  

\column{0.25\textwidth}
\begin{dot2tex}[neato,mathmode,graphstyle={scale=0.7}]
 graph k3two {
    a -- b [color=mLightBrown];
    b -- c [color=mLightBrown];
    a -- c [color=mLightGreen];
  }
\end{dot2tex}  
      
\column{0.25\textwidth}

\begin{dot2tex}[neato,mathmode,graphstyle={scale=0.7}]
 graph k3full {
    a -- b [color=mLightBrown];
    b -- c [color=mLightBrown];
    a -- c [color=mLightBrown];
  }
\end{dot2tex}  

\end{columns}
      
\end{frame}

\begin{frame}[fragile]{Wie viele sollte ich einladen?}


\begin{columns}[T,onlytextwidth]

\column{0.6\textwidth}
\vspace{6mm}

... damit sich 

stets mindestens 3 gegenseitig kennen 

oder 

mindestens 3 gegenseitig nicht kennen?

\vspace{6mm}

\onslide<2->{
Die Ramsey Zahl $R(3,3) = 5$ besagt, dass
für alle $N > 5: \exists$ ein einfarbiger $K_3 \subset K_N$.
}

\vspace{6mm}

\onslide<3->{
\textit{Beweis.}
}

\column{0.4\textwidth}
\begin{center}

\begin{dot2tex}[neato,mathmode,graphstyle={scale=0.7}]
graph k5 {
    a -- b [color=mLightGreen];
    a -- c [color=mLightBrown];
    a -- d [color=mLightBrown];
    a -- e [color=mLightGreen];
    b -- c [color=mLightBrown];
    b -- d [color=mLightGreen];
    b -- e [color=mLightBrown];
    c -- d [color=mLightGreen];
    c -- e [color=mLightGreen];
    d -- e [color=mLightBrown];
}
\end{dot2tex}

\vspace{6mm}

\begin{dot2tex}[neato,mathmode,graphstyle={scale=0.7}]
graph k6 {
    a -- b [color=mLightGreen];
    b -- c [color=mLightBrown];
    c -- d [color=mLightGreen];
    d -- e [color=mLightBrown];
    e -- f  [color=mLightGreen];
    a -- f  [color=mLightBrown];
    a -- c [color=mLightBrown];
    a -- d [color=mLightGreen];
    a -- e [color=mLightGreen];
    b -- d [color=mLightGreen];
    b -- e [color=mLightBrown];
    b -- f [color=mLightGreen];
    c -- e [color=mLightGreen];
    c -- f [color=mLightGreen];
    d -- f [color=mLightBrown];
}
\end{dot2tex}



\end{center}

\end{columns}

\end{frame}

\begin{frame}{Ramsey-Zahlen}

	\begin{definition}[Ramsey-Zahl $R(o, g)$]
		\vspace{0.25cm}
		Man betrachte vollständige Graphen, die in den Farben orange und grün gefärbt ist. Die Ramsey-Zahl $R(o, g)$ ist das kleinste $n$, sodass für alle zwei-gefärbten $K_n$ gilt, dass eine orange Clique (vollständiger Teilgraph) $K_o$ oder eine grüne Clique $K_g$ existiert.
	\end{definition}

\end{frame}


\begin{frame}{Bekannte Ramsey-Zahlen}

Was ist das höchste $k$, sodass $R(k,k)$ bekannt ist?


\begin{itemize}
\item<2-> {$R(4,4) =18$}
\item<3-> {$43 \leq R(5,5) \leq 48$}
\end{itemize}


\onslide<4->{
Warum? Ein Graph mit $m = \binom{43}{2} = 903$ Kanten hat $2^{903}$ mögliche Zwei-Färbungen. 

Die Komplexität der Bestimmung von Ramsey-Zahlen ist unbekannt \cite[p. 13]{Haanpaa2000}.
}

\end{frame}


\begin{frame}{Erdős: Eine untere Schranke}

\onslide<1->{
	\begin{framed}
		\begin{satz}
			Sei $k \geq 2$. Dann gilt $R(k,k) \geq 2^{\frac{k}{2}}$.
		\end{satz}
	\end{framed}

}

\onslide<2->{
\textit{Beweis.}
}

\only<2-3>{
Mithilfe der probabilistischen Methode: Färbe jede Kante zufällig und unabhängig mit gleicher Wahrscheinlichkeit $p_O = p_G = \frac{1}{2}$. 

$\Rightarrow 2^{\binom{N}{2}}$ mögliche Färbungen mit je gleicher Wahrscheinlichkeit $2^{-\binom{N}{2}}$.
}

\only<3-4>{
Nehme zufälligen Teilgraph $A$ der Größe $k$.
\begin{align}
\begin{split}
P(A \text{ ist einfarbig}) & = P(A \text{ ist ganz grün}) + P(A \text{ ist ganz orange})  \\
& = 2 \cdot P\left(\bigcup_{|A_G| = k} A_G\right) \le 2 \cdot \binom{N}{k} \cdot 2^{-\binom{k}{2}}
\end{split}
\end{align}
}

\only<4>{
Für $k \geq 2$ gilt \cite[p. 13]{Aigner2010}
\begin{equation}
\binom{N}{k} \leq \frac{N^k}{2^{k-1}}
\end{equation}
}

\onslide<5->{
Es gilt also $P(A \text{ ist einfarbig}) < 1$ für $N < 2^{\frac{k}{2}}$. 

Das heißt nach der probabilistischen Methode es muss eine Graphenfärbung ohne orangen oder grünen $K_k$ geben.

\hspace*{10mm} $\Rightarrow R(k,k) \geq 2^{\frac{k}{2}}$

\begin{flushright}
$\qed$
\end{flushright}

}

\end{frame}

\begin{frame}{Bekannte Ramsey-Zahlen}

\begin{table}
\centering
\resizebox{\linewidth}{!}{% Resize table to fit within \linewidth horizontally
\input{content/main/ramsey_numbers.tex}
}
\caption{Bekannte Werte oder Grenzen für Ramsey-Zahlen\footnote{https://en.wikipedia.org/wiki/Ramsey\%27s\_theorem}}
\end{table}

\end{frame}





